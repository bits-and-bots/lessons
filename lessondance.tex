\documentclass{lessonplan}

% Constants
\newcommand{\lessonNumber}{Dance}
\title{Dance Dance Revolution}
\author{\linkHome}
\date{}

% Buildup
\setLessonNumber{\lessonNumber}


%Document
\begin{document}

  \maketitle

  \section{Overview}
    \subsection{Summary}
      Flood lights drown the stage as contestants file out from behind
      the curtain\ldots

      Queue: DJ Casper

      As the music starts, the Lego Robot dance party begins!

    \subsection{Objectives}
    \begin{itemize}
      \item Have fun getting the robot to dance!
      \item (Intermediate) Get the robot to dance to the lyrics of the
        Cha Cha Slide.
      \item (Advanced) Get the robot to dance according to user button
        presses.
      \item (Cooperative) Work with the other students to design dance routines.

    \end{itemize}
    \subsection{Key Vocabulary Terms}
    \begin{itemize}
      \item Beat, Motor, Button
      \item (Advanced) Synchronyze
    \end{itemize}
    \subsection{Materials}
      \begin{itemize}
        \item Any materials needed, including the standard set of
          equipment
        \item Standard ev3 brick
        \item Laptops
        \item Programming Cables
        \item Smart phones with recordings of songs and bluetooth speaker
        \item Demo Program: dancing to a pop song

      \end{itemize}
  \section{Procedure}
    \subsection{Setup (5 minutes)}
      Create a big space for the robots to dance!

      Turn on computers and open the programming environment.
      Build and turn on bots.
    \subsection{Introduction (5 minutes)}
      Introduce with a video of a musician that is recognizable to the
      students. A video with a famous dance (like Thriller), might
      be a fun way to start it.

      Run the sample program you created to show an example of a
      robot dancing to the music.

      After showing your video, have some fun introducing the
      dance party. Consider something along the lines of: ``Hello everybody, we are here
      to find the next background dancers for our new hit band
      Electric Grooves! The audition song will be the Cha Cha Slide,
      you have 45 minutes to prepare your routines." 

    \subsection{Guided Practice}
      Start playing the song (on an endless loop) that the students
      will be making programs to dance to.
      
      \subsubsection{Useful Tips:}
      \begin{itemize}
        \item The main tools people will use are move blocks:
          Here is a picture of the Move Steering block:

          \procedureImage[0.6]{dance-1.png}
          \textcolor{magenta}{Magenta}: Method of steering control \hfill \\
          \textcolor{red}{Red}: Steering  \hfill \\
          \textcolor{blue}{Blue}: Power  \hfill \\
          \textcolor{green}{Green}: Rotations  \hfill \\
          \textcolor{orange}{Orange}: Brake at end or not  \hfill \\

        \item People may also want to use buttons to input dance directions:
          \procedureImage[0.3]{dance-2.png}

      \end{itemize}
    \subsection{Extensions}
      \begin{itemize}
        \item Make two robots do a partner dance (advanced).
        \item Do the same routine to the same song played twice as fast.
      \end{itemize}
    \subsection{Conclusion and Evaluation}
      Use the last few minutes of class to run the dance party:

      ``Times up everybody! Let's see the robots dance, while our special guest Electic Grooves
      picks out their new dancers!"

      \subsubsection{Exit Questions}
      \begin{itemize}
        \item How does your dance correspond to the beats of the song?
      \end{itemize}

\end{document}
