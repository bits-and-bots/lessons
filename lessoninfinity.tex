\documentclass{lessonplan}

% Constants
\newcommand{\lessonNumber}{Infinity}
\title{The Infinity Lesson}
\author{\linkHome}
\date{}

% Buildup
\setLessonNumber{\lessonNumber}


%Document
\begin{document}

  \maketitle

  \section{Overview}
    \subsection{Summary}
      This lesson teaches students about infinite vs. finite, discrete vs. continuous, (self-similar) fractals, 
      tessellations, and vector vs. pixel art.

      It will also give students the basis to understand sequences, series, mapping, and proof by contradiction. 

      Students will be introduced to powerful design technology, get a chance to hone their fine motor skills, and 
      even get to take home their creations.

      This lesson can, and should, be run differently every time depending on what students are curious and excited
      about, and what materials are available.
    \subsection{Objectives}
    \begin{itemize}
      \item Think about infinity (big infinity and little infinity) and how the real world is finite.
      \item Cut and fold paper into repeating patterns.
      \item (Intermediate) Think about continuity and how the real world is discrete.
      \item (Intermediate) Think about how pixels are discrete and vectors are continuous, and how technology can be 
        used to model continuity with vectors. Make vector and pixel art.
      \item (Advanced) Reason about properties of little and big infinity.
      \item (Advanced) Reason about infinities of different sizes.
      \item (Cooperative) Create collaborative fractals and/or tessellations.
    \end{itemize}
    \subsection{Key Vocabulary Terms}
    \begin{itemize}
      \item infinite, infinitesimal, finite, fractal, tessellation, translation, rotation, reflection
      \item (Intermediate) discrete, continuous
      \item (Advanced) vector, pixel, (sequence/pattern/repetition) depth
    \end{itemize}
    \subsection{Materials}
      \begin{itemize}
        \item White letter paper -- 1 ream
        \item Color letter paper -- 1 ream, or enough sheets for each student and instructor to get 1
        \item Tissue paper -- 1 sheet
        \item Scissors -- a few pairs; if not enough for every student, they can share
        \item Tape -- 1 or 2 rolls, color/opacity doesn't matter
        \item Some sort of writing implement (pencils, pens, markers, etc.) for each student
        \item Origami paper -- at least 1 sheet per intermediate student and instructor
        \item Origami hydrangea fractals and herringbone tessellations of different depths, created by Verda, shown 
          below for instructor reference during preparation for the lesson. The originals should be brought to the 
          lesson, not just these images.
          \procedureImage[0.1]{example-hydrangea.jpg}
          \procedureImage[0.1]{example-herringbone.jpg}
        \item iPad (alternative: computer with Adobe Photoshop and Adobe Illustrator installed)
        \item (on iPad) Adobe Fresco document with the same layer in vector and pixel format, with large enough 
          dimensions that the vector and pixel layers look the same (or as close as possible) at 100\% zoom.
          Alternative if no iPad: same layer in Photoshop and Illustrator, but one is pixel and one is vector.
        \item (on iPad) Adobe Fresco document with Sierpinski triangle. Alternative if no iPad: Illustrator document.
        \item (on iPad) Adobe Fresco document with triangle tessellation. Alternative if no iPad: Illustrator document.
        \item (on iPad or computer) YouTube Video 1: 
          \url{https://www.youtube.com/watch?v=4YDHsMUQbVg}
        \item (on iPad or computer) YouTube Video 2: 
          \url{https://www.youtube.com/watch?v=Ms655fMOxWE&list=PLUx6ByGW00PzyFMhIoL47_ZezTEQpquh1&index=1}
        \item (on iPad or computer) YouTube Video 3: 
          \url{https://www.youtube.com/watch?v=7bFPoGRUJqc&list=PLUx6ByGW00PzyFMhIoL47_ZezTEQpquh1&index=2}
        \item (on iPad or computer) YouTube Video 4: 
          \url{https://www.youtube.com/watch?v=U7TKtjQl1zk&list=PLUx6ByGW00PzyFMhIoL47_ZezTEQpquh1&index=3}
        \item (on iPad or computer) YouTube Video 5: \url{https://www.youtube.com/watch?v=RDN7JZpXor4}
        \item (on iPad or computer) YouTube Video 6: \url{https://www.youtube.com/watch?v=nw5RLvN7fYA}
      \end{itemize}
  \section{Procedure}
    \subsection{Setup (10 minutes)}
      Cut a few sheets of white letter paper into small (about 3x3in) pieces, so there are enough for each student and 
      each instructor to have 3 pieces. Distribute 5 full sheets of white letter paper, 1 full sheet of color paper,  
      and 3 small pieces of white letter paper to each student. Pass around writing implements, scissors, and tape.
    \subsection{Introduction (10 minutes)}
      ``Today we're going to talk about infinity.''

      Take a sheet of white letter paper and fold it in half. 

      ``How many times can you fold a piece of paper?''

      Ask each student to take 1 sheet of their white letter paper and fold it as many times as they can.
      Go around and ask how many folds. Give them your answer -- should be around 7. 

      Take the sheet of tissue paper and fold it as many times as you can. Should be more than 7. This shows how the 
      thinner a paper is compared to its area, the more times it can be folded.

      Pick up a fresh piece of white letter paper.
      \subsubsection{Introduce little infinity}
        ``Imagine a piece of paper so thin that you could fold it a hundred times. Imagine a piece of paper so thin that 
        you could fold it a million times. Now, imagine a piece of paper so thin that you could keep folding it forever. 
        That's little infinity. We would say that that paper is infinitely thin, or that its thickness is infinitesimal, 
        meaning infinitely small.''

        Pause for questions.
      \subsubsection{Introduce big infinity}
        ``Now let's think about the opposite, things that are really big. Imagine that we have a piece of paper that's 
        this thick'' (hold up letter paper) ``but it's the size of this table. What about a piece of paper the size of 
        this room? This city? This galaxy? The biggest thing you could possibly imagine? All of those papers would 
        eventually run out of folding space, because even though they're really big, they still have limits. Now imagine 
        a piece of paper that's this thick but goes on forever, without limits.'' (gesture to denote spreading out 
        radially) ``That's big infinity. We would say that that paper is infinitely vast/wide, or that its area is 
        infinite.''
      \subsubsection{Introduce finitude}
        ``If something isn't infinite or infinitesimal, it's finite. It has clearly defined limits. We can see the limits 
        of this paper right here:'' (hold up letter paper and gesture to its edges and its thickness) ``Everything in the 
        real world is finite -- if we want infinity, we look to the world of math. Sometimes it's really useful to think 
        about, `what if we took this finite pattern and kept repeating it forever?'"
      \subsubsection{Introduce today's activities}
        ``Today we're going to make a few patterns and think about what it might look like if we kept repeating them 
        forever. The two types of patterns we're going to make are called fractals and tessellations.''

        Show triangle translation tessellation in Fresco.

        ``Tessellations are a type of repeating pattern where elements of the pattern are next to each other.''

        Show rotation and reflection tessellations in Fresco.

        ``The elements might stay the same when they're repeated, or they might be rotated or flipped.''

        Show Sierpinski triangle in Fresco.

        ``Fractals are a type of repeating pattern where elements of the pattern are within each other. This particular 
        fractal is called the Sierpinski triangle, named after a mathematician named Sierpinski.''

        ``Both fractals and tessellations can go on repeating forever, or we can cut them off at a limit we choose. If 
        we cut them off at a limit, the number of repetitions is called the depth of the pattern. If we never cut it 
        off, the depth would be big infinity.''
    \subsection{Demonstration and Guided Practice}
      It might be helpful to split up into two groups here -- beginner and intermediate -- so the beginner students can
      take their time and intermediate students can have a bit more time to work on the more complex art and math 
      concepts in the Extensions section.
      \subsubsection{Fractals}
        Show them YouTube Video 1, demonstrating making a Sierpinski triangle with cuts in a piece of paper.
        Follow along yourself, and have the students do the same with their second sheet of white letter paper. Pause at 
        every step so all students can catch up. At the very end, have the students attach their color paper to the back 
        of their creations.
      \subsubsection{Tessellations}
        ``In tessellations, as opposed to fractals, elements of the pattern are repeated next to each other, not inside 
        one another. The elements might stay the same when they're repeated, or they might be rotated or flipped.''
      
        Show triangle tessellation in Fresco.
      
        Show them YouTube Video 2, demonstrating making a tessellation by translation. Follow along yourself, and have 
        the students do the same. Pause at every step so everyone can catch up. Label your creation ``TRANSLATION'' and 
        have the students do the same. Trace tessellation element across a whole sheet of white letter paper.

        Show them YouTube Video 3, demonstrating making a tessellation by reflection. Follow along yourself, and have 
        the students do the same. Pause at every step so everyone can catch up. Label your creation ``REFLECTION'' and 
        have the students do the same. Trace tessellation element across a whole sheet of white letter paper.

        Show them YouTube Video 4, demonstrating making a tessellation by rotation. Follow along yourself, and have 
        the students do the same. Pause at every step so everyone can catch up. Label your creation ``ROTATION'' and 
        have the students do the same. Trace tessellation element across a whole sheet of white letter paper.
    \subsection{Extensions}
      \subsubsection{Collaborative fractal drawing}
        One student draw a big shape on a sheet of paper, another student draw another shape inside it, another draw 
        a third inside the second, and then draw the first big shape again but smaller within the third shape. All 
        students keep going with the repetition as small as they can. 
      \subsubsection{Collaborative tessellation creation}
        Students form a group and make the biggest tessellation they can together! Make a lot of copies of the same 
        tessellation element (by tracing and cutting copies of the original element), and set them up all over the 
        room as far as they can go.
      \subsubsection{More complex origami fractals and tessellations}
        Show off Verda's example origami hydrangeas and herringbone tessellations, as well as YouTube videos 
        demonstrating how to make them (YouTube Videos 5, 6); give students origami paper and let them try their 
        hand at making the origami fractals/tessellations to whatever depth they want.
      \subsubsection{(Intermediate) Discrete vs. continuous}
      \subsubsection{(Intermediate) Vector vs. pixel art}
        Talk about how pixel art is discrete because pixels are distinct from each other and there is clear separation 
        between them. When you zoom in enough, you can see where one pixel ends and the next one begins. Vector art, 
        on the other hand, is continuous. You can keep zooming in (theoretically; Fresco actually limits how far you
        can zoom) and still see a smooth line. 

        Let the kids play around with Fresco. Make sure they're being gentle with the iPad. Talk about how this really
        cool technology is created and maintained by software designers and developers. Answer questions about software 
        design and development to the best of your ability. 
      \subsubsection{(Advanced) Reasoning about infinity}
        ``This paper'' (show letter papers, unfolded and folded) ``runs out of folding space because each time it's 
        folded, it gets thicker. With the infinitely thin paper, it would have to be the case that when it's folded, 
        it stays infinitely thin.''

        Why?

        ``Imagine the infinitely thin paper got thicker when you folded it. It would eventually get as thick as this 
        paper, and by that time, its area would be a lot smaller. That means it would eventually run out of space, 
        and we wouldn't be able to fold it anymore. If we can't fold it anymore, that means it's not infinite. This 
        tells us that infinite folds means the paper doesn't get any thicker with each fold.''
      \subsubsection{(Advanced) Reasoning about infinities of different sizes}
        Talk about the Infinite Hotel Paradox 
        (\url{https://en.wikipedia.org/wiki/Hilbert%27s_paradox_of_the_Grand_Hotel}). Consider finitely many new 
        guests or infinitely many new guests.

        Same situation as paper not getting any thicker with each fold: x = 2x if x is infinite or infinitesimal. 
        Doesn't have to be 2: x = nx where n element of N, if x is infinite or infinitesimal. 

        2 infinities are the same size if there's a 1:1 correspondence; if we can pair each element of Infinity 1 to
        a different element of Infinity 2, and there's none left over in Infinity 2. The set of natural numbers, or 
        ``counting numbers'', is infinite -- you can keep adding 1 and getting a new number forever. We can use this 
        infinite set to measure the size of other infinities. If we can count each element in another infinite set 
        using the natural numbers, then it's the same size as N. If there are other numbers that can't be counted
        using N, then it's a bigger infinity than N. 

        Draw infinities (N, Q, R) on paper to help kids visualize the process of counting them.
      \url{https://www.businessinsider.com/the-different-sizes-of-infinity-2013-11}
    \subsection{Conclusion and Evaluation}
      Give the kids a chance to show the fractals and tessellations that they created.
      \subsubsection{Exit Questions}
      \begin{itemize}
        \item What's another word to describe little infinity? (Hint: ``little infinity'' is a noun, but this word is an 
          adjective)
        \item What's another word to describe big infinity?
        \item What do you call something that isn't infinitesimal or infinite?
        \item Out of infinite, infinitesimal, and finite, which one(s) exist(s) in the real world?
        \item What's the difference between a fractal and a tessellation?
        \item What is the depth of the fractal you created? 
        \item (Intermediate) What's the difference between discrete and continuous? 
        \item (Intermediate) What's the difference between vector and pixel art?
      \end{itemize}

\end{document}
