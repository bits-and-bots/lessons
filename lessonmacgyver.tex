\documentclass{lessonplan}

% Constants
\newcommand{\lessonNumber}{MacGyver}
\title{\lessonNumber Lesson: Escape the Island}
\author{\linkHome}
\date{}

% Buildup
\setLessonNumber{\lessonNumber}


%Document
\begin{document}

  \maketitle

  \section{Overview}
    \subsection{Summary}
      This lesson plans to teach general problem solving using unconventional
      tools as well as chaining events together to achieve one end. Kids will
      try to capture a pet crab (built out of lego crab) with the remains
      of a shipwreck.

    \subsection{Objectives}
    \begin{itemize}
      \item Event chaining (how one event affects the next),
      general robustness/engineering fun
      \item (Intermediate) Conservation of energy, collisions and transfer
      of energy
      \item (Advanced) Potential vs kinetic energy
      \item (Cooperative) How to work on different pieces of the same system
      and tie them together in the end
    \end{itemize}

    \subsection{Key Vocabulary Terms}
    \begin{itemize}
      \item Engineering
      \item (Intermediate) Conservation of energy, energy
      \item (Advanced) Potential, kinetic energy
    \end{itemize}

    \subsection{Materials}
      \begin{itemize}
        \item Cardboard box
		\item String
		\item Tape
		\item Netting
		\item Erasers
		\item Balls (marbles, rubber balls, etc)
		\item Pencils
		\item Cereal boxes
		\item Wedge
		\item Cardboard tube
		\item Waterbottle
		\item And anything else that would make this lesson fun!
      \end{itemize}

  \section{Procedure}
    \subsection{Setup (10 minutes)}
	\paragraph{}
      Place the lego crab on the ground, and tape a square around it. Box should be large enough so that crab is fully enclosed, and has a few inches
	of space around it. 
	Place materials in piles around the crab.

    \subsection{Introduction (5 minutes)}
      Introduce using a video of a cool Rube Goldberg Machine (we suggest
      \href{https://www.youtube.com/watch?v=qybUFnY7Y8w}). This is a fun
      music video, so if you can set up sound, it's all the better.

      Now introduce the scenario. For example:
      "After 3 days stuck at sea, you castaways land on a deserted island.
      All that's left on the wrecked boat is a fishing line (hold up the string),
      an empty water bottle (which they can fill), a cardboard box (chest from
      their pirate treasure), some erasers, a wedge and a board,
      a cardboard tube, a cereal box, and a ball.''

      Explain to the kids that getting too close to the crab spooks it.
      They must make a machine to capture it without standing next to it.

    \subsection{Demonstration}
      Show the ball hit the erasers, so they fall over like dominoes.

    \subsection{Guided Practice}
		\begin{itemize}
			\item Show that the box can serve as a trapping mechanism to capture the crab
			\item If we have a net, show that we can trap the crab in the net
			\item Show that a tube can be used to roll a ball into domino-erasers, which can hit a wedge, tipping 
					a waterbottle and knocking into a cereal box ...
			\item Basically, show different combinations of how to use the materials
			\item Use tape at some point
			\item Point out where energy transfer happens
		\end{itemize}

      \subsubsection{sticking points:}
      \begin{itemize}
        \item A box can be held up with a ruler or a string - there's more than one way to capture the crab with the box
	    \item Encourage them to compare and contrast ideas, rather than choosing one and sticking with it
		\item Encourage students to keep trying, even if their ideas fail at first
		\item Point out places in their own designs where energy is transferred
      \end{itemize}
    \subsection{Extensions}
      \begin{itemize}
        \item Try to use as few materials as possible
		\item Try to use all the materials given to them
		\item Try to add in one more material
		\item Try to remove one material
      \end{itemize}
    \subsection{Conclusion and Evaluation}
      Present the traps to get the crab! 
      \subsubsection{Exit Questions}
      \begin{itemize}
        \item How does event chaining work?
		\item How is energy transferred in collisions?
      \end{itemize}

\end{document}
