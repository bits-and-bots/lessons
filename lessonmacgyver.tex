\documentclass{lessonplan}

% Constants
\newcommand{\lessonNumber}{MacGyver}
\title{\lessonNumber Lesson: Escape the Island}
\author{\linkHome}
\date{}

% Buildup
\setLessonNumber{\lessonNumber}


%Document
\begin{document}

  \maketitle

  \section{Overview}
    \subsection{Summary}
      This lesson plans to teach general problem solving using unconventional
      tools as well as chaining events together to achieve one end. Kids will
      try to capture a pet crab (built out of lego crab) with the remains
      of a shipwreck.

    \subsection{Objectives}
    \begin{itemize}
      \item Event chaining (how one event affects the next),
      general robustness/engineering fun
      \item (Intermediate) Conservation of energy, collisions and transfer
      of energy
      \item (Advanced) Potential vs kinetic energy
      \item (Cooperative) How to work on different pieces of the same system
      and tie them together in the end
    \end{itemize}

    \subsection{Key Vocabulary Terms}
    \begin{itemize}
      \item Engineering
      \item (Intermediate) Conservation of energy, energy
      \item (Advanced) Potential, kinetic energy
    \end{itemize}

    \subsection{Materials}
        % TODO: find materials
      \begin{itemize}
        \item Any materials needed, including the standard set of
          equipment
        \item Standard ev3 brick
        \item Laptops
        \item Programming Cables
        \item Smart phones with recordings of songs
        \item Demo Program: pop song played on the robot using sound
          blocks
      \end{itemize}

  \section{Procedure}
    \subsection{Setup (10 minutes)}
        % TODO: setup, add dimensions for square

      Place the lego crab on the ground, and tape a square around it.

      \paragraph{}
      Nothing Special, setup robots and computers.
    \subsection{Introduction (5 minutes)}
      Introduce using a video of a cool Rube Goldberg Machine (we suggest
      \href{https://www.youtube.com/watch?v=qybUFnY7Y8w}). This is a fun
      music video, so if you can set up sound, it's all the better.

      Now introduce the scenario. For example:
      ``After 3 days stuck at sea, you castaways land on a deserted island.
      All that's left on the wrecked boat is a fishing line (hold up the string),
      an empty water bottle (which they can fill), a cardboard box (chest from
      their pirate treasure), some erasers, a wedge and a board,
      a cardboard tube, a cereal box, and a ball.''

      Explain to the kids that getting too close to the crab spooks it.
      They must make a machine to capture it without standing next to it.

    \subsection{Demonstration}
      Show the ball hit the erasers, so they fall over like dominoes.

    \subsection{Guided Practice}

      % TODO: formalize this into guided practice.
      % Tube can be used to roll the ball into the erasers, which fall
      % like dominoes until hitting the wedge, which tips over the water bottle
      % (now filled) knocking into the cereal box which is holding up the larger
      % box. The larger box falls on the crab.

      \subsubsection{sticking points:}
      \begin{itemize}
        \item Students can use the arrow keys to scroll the program
        \item It may be easier to work on parts of the song
          indiviually
        \item Make sure the robot is set to wait until completion
          before playing the next tone
        \item Copy and Paste is a good way to preserve the correct
          settings easily
      \end{itemize}
    \subsection{Extensions}
      \begin{itemize}
        \item Identify parts of the song that loop, and put them in
          loops
        \item play "Mary had a little lamb", have the students find
          the right pitch for the fifth through experimentation
        \item When multiple groups have the song correctly made, have
          them play the program in a round
        \item Do the same challenge one octave up
      \end{itemize}
    \subsection{Conclusion and Evaluation}
      Give the kids a chance to show the music that they created
      \subsubsection{Exit Questions}
      \begin{itemize}
        \item How does freqency affect the pitch of a tone?
      \end{itemize}

\end{document}
