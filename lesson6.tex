\documentclass{lessonplan}

\title{Lesson 6: Musical Robots}
\author{\linkHome}
\date{}

\begin{document}

  \maketitle

  \section{Overview}
    \subsection{Summary}
      This lesson teaches students how to synthesize music using
      robots.  For more advanced students, it delves into the
      structure of sound, and the ways we can measure it.
    \subsection{Objectives}
    \begin{itemize}
      \item Differentiate between pitches
      \item Sequence sounds and rests into a tune
      \item (Intermediate) Identify sequences that repeat in a song
        and use loops to program them efficiently
      \item (Intermediate) learn how Hz relates to a sound's pitch
      \item (Advanced) Understand what Hz is, and be able to relate it
        to the duration of a track played on a loop.
      \item (Cooperative) Play a song in a round
    \end{itemize}
    \subsection{Key Vocabulary Terms}
    \begin{itemize}
      \item Pitch, Higher Pitch, Lower Pitch
      \item (Intermediate) Hertz, Frequency
      \item (Advanced) Period
    \end{itemize}
    \subsection{Materials}
      \begin{itemize}
        \item Standard ev3 brick
        \item Laptops
        \item Programming Cables
        \item Smart phones with recordings of songs
        \item Demo Program: pop song played on the robot using sound
          blocks
      \end{itemize}
  \section{Procedure}
    \subsection{Setup (10 minutes)}
      \paragraph{}
      Nothing Special, setup robots and computers.
    \subsection{Introduction (5 minutes)}
      Introduce with a video of a musician that is recognizable to the
      students, preferring one that makes heavy use of synthesizers.
      Explain that the sounds are made by computers and that the robot
      is capable of making similar music (it's not entirely untrue).
    \subsection{Demonstration}
      Play the demo program, with sound blocks using the pitch
      setting.  Then show them how to make the notes 1, 2 and 3 of the
      major scale using the sound blocks, in addition to the wait
      block.  Make a program that plays the first 3 notes of "three
      blind mice", with the rest at the end, leave the notes on the
      projector so the students know which frequencies to use.
    \subsection{Guided Practice}
      Play a recording of "Three blind mice" and have them try to
      synthesize it using the robots
      \subsubsection{sticking points:}
      \begin{itemize}
        \item Students can use the arrow keys to scroll the program
        \item It may be easier to work on parts of the song
          individually
        \item Make sure the robot is set to wait until completion
          before playing the next tone
        \item Copy and Paste is a good way to preserve the correct
          settings easily
      \end{itemize}
    \subsection{Extensions}
      \begin{itemize}
        \item Identify parts of the song that loop, and put them in
          loops
        \item play "Mary had a little lamb", have the students find
          the right pitch for the fifth through experimentation
        \item When multiple groups have the song correctly made, have
          them play the program in a round
        \item Do the same challenge one octave up
      \end{itemize}
    \subsection{Conclusion and Evaluation}
      Give the kids a chance to show the music that they created
      \subsubsection{Exit Questions}
      \begin{itemize}
        \item How does freqency affect the pitch of a tone?
      \end{itemize}

\end{document}
