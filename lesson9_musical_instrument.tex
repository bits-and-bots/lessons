\documentclass{lessonplan}

% Constants
\newcommand{\lessonNumber}{}
\title{Lesson \lessonNumber: [Lesson Title]}
\author{\linkHome}
\date{}

% Buildup
\setLessonNumber{\lessonNumber}


%Document
\begin{document}

  \maketitle

  \section{Overview}
    \subsection{Summary}
      The goal of this lesson is to introduce students to the technological nature of instruments,
       and how different instruments create the tones they do. 
    \subsection{Objectives}
    \begin{itemize}
      \item Go over the basics of freqency and waves through an demonstration of vibrating strings
      \item (Intermediate) Show a wider variety of methods to generate sounds 
      including tubes and pipes (used in trumpets), as well as lengths of metal 
      (xyolophone) and drum sizes.
      \item (Advanced) Tie all instruments together about distances of travel, 
      whether realted to string length or pipe length, or drum size.
      \item (Cooperative) Allow students to create their own instrument designs, 
      then work with basic materials to build different tones. 

    \end{itemize}
    \subsection{Key Vocabulary Terms}
    \begin{itemize}
      \item List some of the key takeaways from the lesson.  These are
        also good things to ask about at the conclusion of the lesson
      \item Pitch, Higher Pitch, Lower Pitch
      \item (Intermediate) Hertz, Frequency, Wavelength
      \item (Advanced) Period

    \end{itemize}
    \subsection{Materials}
      \begin{itemize}
        \item Guitar String & Twine
        \item Scissors
        \item Metal Slats of various lengths
        \item Table Clamps
        \item Measuring Tape
        \item Lengths of small PVC pipes
        \item Smart phones with recordings of songs
        \item Plastic Ukelele
        \item Smart-phone with slow motion camera (or video for reference). Connect phone camera to projector.
        \item Air blower
      \end{itemize}
  \section{Procedure}
    \subsection{Setup (10 minutes)}

      \paragraph{}
            If any of the teachers play an isntrument, have them bring them in (if applicable), 
      to start the lesson off with a beat! 

    \subsection{Introduction & Demonstration (5 minutes)}
      Start the lesson by playing brought-in instruments to the students to get them interetsed
       and engaged with the concept of music for the lesson. If possible,
        point out the key differences and similarities
        in the different instruments as a hint of what's to come.
    \subsection{String Frequency}
      Bring out different lengths of strings attached at different points by screws in wood. Ask the kids 
      to flick the strings, and determine how the different levels of tighness affect the tighness. 
      The kids may notice about how the string moves, ask them to predict how it changes based on how tight it is.
      Use a phone camera to take a slow motion camera of each of the different string variations, showing the vibrations.


    \subsection{Wavelength & Natural Frequency}

      Clamp various metal strips at the edge of a table, with measured lengths sticking free past the edge. 

      \subsubsection{core points:}
      \begin{itemize}
        \item Students can flick the metal strips to hear the vibrations
        \item Different lengths will produce different tones based on wavelengths
        \item Vibrations can also be viewed using slow motion cameras to slow similarities
      \end{itemize}

    \subsection{Wind Instruments}

      Various lengths of PVC tubing

      \subsubsection{core points:}
      \begin{itemize}
        \item Use air-blower at the edge of the PVC tubes to produce sounds
        \item Note the difference in pitch based on the length/size of the PVC tube
      \end{itemize}
    \subsection{Extensions}
      \begin{itemize}
        \item 
      \end{itemize}
    \subsection{Conclusion and Evaluation}
      Use the existing types of instruments to make music!
      \subsubsection{Exit Questions}
      \begin{itemize}
        \item How does freqency affect the pitch of a tone?
      \end{itemize}

\end{document}
