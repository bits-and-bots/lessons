\documentclass{lessonplan}

% Constants
\newcommand{\lessonNumber}{9}
\title{Lesson \lessonNumber: [Lesson Title]}
\author{\linkHome}
\date{}

% Buildup
\setLessonNumber{\lessonNumber}


%Document
\begin{document}

  \maketitle

  \section{Overview}
    \subsection{Summary}
      Here we give an overview of the Lesson and what it teaches, to
      both younger and older audiences

      This lesson teaches students how to crack and create their own
      ciphers through the context of an ancient treasure.

      For more advanced students, this lesson goes into basic mathematics 
      of prime number and modulo based ciphers. 

    \subsection{Objectives}
    \begin{itemize}
      \item Introduce students to basic letter ciphers
      \item (Intermediate) Introduce the concept of morse code and modern cyber security.
      \item (Advanced) Describe the concept of prime number factorization for use in security systems.
      \item (Cooperative) Have separate groups work together to help solve an ancient cipher!
    \end{itemize}
    \subsection{Key Vocabulary Terms}
    \begin{itemize}
      \item List some of the key takeaways from the lesson.  These are
        also good things to ask about at the conclusion of the lesson
      \item Cipher, Encryption, Decryption
      \item (Intermediate) Cyber-security, Modulo
      \item (Advanced) Prime-number factorization, RSA
    \end{itemize}
    \subsection{Materials}
      \begin{itemize}
        \item 2x Chromebooks for the group to work with for the Morse Code portion.
        \item Cipher Wheel
        \item 2x Tubes of certain diameters (One works with Scytale), hole drilled at end and tied to wooden base.
        \item Alien language chart
        \item 2x Morse chart
        \item 2x USB with morse data on it (labeled 'part-1' and 'part-2', both also have morse code chart png).
        \item 2x computer with ability to play sound and take USB
        \item 2x treasure box (first and last puzzles)
        \item Black light pen/light
      \end{itemize}
  \section{Procedure}
    \subsection{Setup (10 minutes)}
      \begin{itemize}
        \item Place cipher wheel and caesarian cipher code in first box, labelled '1'.
        \item Place Scytale strips in location revealed by caesarian cipher
        \item Place scytale bases in different locations, revealed by notes in Box 1
        \item Scytales produce 2-part descriptions of where next clue is (depending on location).
            Examples could be 'opposite' 'window'. Place USBs at those locations.
        \item Lock must be set to work with decided on number code (from Morse).
        \item Place "Gold Coins" chocolate in the second box, labelled '2'. 
        \item Stick a piece of paper that has written on it "?->A", where ? is the key for the first cipher.
      \end{itemize}
      \paragraph{}
      Setup two tables with computers on but not used initially. 
      Place various supplies in a box on each table.
      \paragraph{}

    \subsection{Introduction (5 minutes)}
      The kids have stumbled upon a great and exciting adventure! 
      While exploring a ruin of an abandoned mansion, 
      you have all entered an large room filled with many items, 
      and in the center are two treasure chests. 

      One of them is labeled 1 and another labeled 2. I guess we should try and open the 1 box first! 
      \par
      Is there anything on the chest anywhere?

    \subsection{Substitution Cipher}
      Show kids paper with 'cipher hint' that was 'found' nearby.
       Go through it with a second copy and help them as needed.

    \subsection{Caesarian Cipher}

      Once the kids have unlocked the first chest, they will find the cipher wheel and the code.
      The code word is written on the back of the wheel in black light.

      Using that code work (usually 3-4 letters long), use the wheel to uncover the next hint location.
      
      \begin{itemize}
        \item If the kids are having a tough time getting past any parts of this, help them out as necessary
        \item An example caesarian cipher can be provided on the board via a 'found' slip of paper to help out.
      \end{itemize}
      
    \subsection{Scytale}

      The cipher wheel will provide a hint to a location where the scytale strips are.
      They will have a hint on the back of them (a drawing of the tubes, and them wrapping around them).

      Let the kids figure out how to use the tubes, but then provide hints like before as necessary.

      The kids will have a variety of tubes (available from the start), all of different size diameters.
      The correct size tube will reveal the location of one of the two USBs hidden in the room
      (if the room is bare, bring sticky wall adornments and place USBs behind them).

    \subsection{Morse Code}

      The USBs found through the Scytale, can be plugged into the chrome books.

      The USB will have the following file structure: 

      \procedureImage[0.6]{usb_file_structure1.PNG}

      The morse chart will contain a reference for the kids to translate the code found in the mp3 file.

      \procedureImage[0.6]{morse_chart.PNG}

      The mp3 file will contain a letter description of the numbers required to open the chest2 padlock.


      The second USB will contain two files, a rosetta stone, and a hidden message. 
      The rosetta stone contains two identical messages, in both english and the secret language. 
      This can be used to build a translator and then eventually solve the hidden message.
      The hidden message will contains the second part of the 4 digit code to open the final chest.

      \procedureImage[0.6]{hidden_language_translator.PNG}

      Both USBs will need to be separately translated, then the results can be used to open the final chest!
      The chest will contain some form of reward for the children, and will prompt the post-lesson pizza.
      

    \subsection{Extensions}
      \begin{itemize}
        \item Describe prime factorization (take number and the code would be the prime factors)
      \end{itemize}
    \subsection{Conclusion and Evaluation}
      Ask the kids about all of the types of ciphers, and ask about their favorites.
      \subsubsection{Exit Questions}
      \begin{itemize}
        \item What was the easiest cipher to crack?
        \item Could you prepare to solve those ciphers faster?
        \item Can you make up your own cipher?
      \end{itemize}

\end{document}
