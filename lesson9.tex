\documentclass{lessonplan}

% Constants
\newcommand{\lessonNumber}{}
\title{Lesson \lessonNumber: The B\&B Stock Exchange Game}
\author{\linkHome}
\date{}

% Buildup
\setLessonNumber{\lessonNumber}


%Document
\begin{document}

  \maketitle

  \section{Overview}
    \subsection{Summary}
      This lesson will be a simulation of a stock exchange on a tropical island named Bot Island. 
      We will teach student what stocks are, how they work, and how they are influenced by real-world events.
      For more advanced students, it delves into how the stocks make money (through compound interest and dividends)
      and what are the best strategies for making the most money.

    \subsection{Objectives}
    \begin{itemize}
      \item Learn what a stock is, and how to buy and sell stocks
      \item (Intermediate) Learn about compound interest, and how money grows when you leave it in 
                an account that gains interest. 
      \item (Advanced) Learn different strategies of how to handle money regarding the market (such as 
               diversification, investing in hot-item stocks, avoiding the stock market altogether, etc), and discuss 
               different kinds of ways to diversify a portfolio (including but not limited to bonds, index funds, 
               mutual funds, and ETFs). Also learn how to examine and rate a stock by using information like balance sheets P/E ratios.
      \item (Cooperative) Students can create their own "firms" by grouping up and discussing how they would buy or sell their stocks.
    \end{itemize}
    \subsection{Key Vocabulary Terms}
    \begin{itemize}
      \item Stocks, buy/sell (in context of the market)
      \item (Intermediate) Compound Interest
      \item (Advanced) Portfolio, diversification, bonds, index funds, mutual funds, ETFs, balance sheet, P/E ratios
    \end{itemize}
    \subsection{Materials}
      \begin{itemize}
        \item Instructor Laptop 
        \item Pens and paper 
        \item "BotCash" - papers printed with the B\&B logo to act as currency
        \item Papers printed as forms to declare buys/sells
        \item Prizes for students to exchange for their BotCash, ranging from lesser-value items like pencils and erasers to larger-value items
                 like pencil cases, sketch books, and pen or highlighter sets
      \end{itemize}
  \section{Procedure}
    \subsection{Setup (10 minutes)}
     Prepare instructor laptop with slideshow containing information about what a stock is, and how you can buy and sell stocks.
     Another slideshow should follow explaining the premise of Bot Island, and include the fake newspaper screenshots from 
     Bot Island explaining all the current events that occur on Bot Island. Instructors discuss amongst themselves what
     strategies they will employ in investing their money. One should invest in all stocks equally (simulating an index fund),
     another should invest all their money in one stock, and (optionally) another can refuse to put their money into the market.
    \subsection{Introduction (5 minutes)}
     Present the first slideshow, which will explain what a stock is, and how a stock exchange consists of all the available stocks
     in a certain place. Then, explain how we are all new residents of the tropical island called Bot Island, and how there
     are several companies on the island that each offer stocks. Give each student their set amount of initial BotCash,
     and ask them if they'd like to participate in the island's stock exchange (appropriately named the B\&B exchange). 

    \subsection{Demonstration}
     Instructors will also be given their BotCash, and will announce what their strategies are for investing into the stock
     market (as listed above). They should demonstrate how to submit their form to purchase their stocks.
     An instructor will then update a pre-made spreadsheet detailing the transactions.
    \subsection{Guided Practice}
      Inform students that they are allowed to form groups to make their own "firm", and can discuss amongst themselves
      which stocks they would like to purchase or sell. Have them submit their forms, and the instructor should update
      a spreadsheet detailing the transactions that the students made. After all students have made their purchases,
      the first "round", or the first newspaper, should be shown with the first event. Events should start simply first,
      perhaps a nice sunny day on the island. These events will be shown as the front pages of newspapers in the
      slideshow. As students are starting to feel more comfortable making decisions and buying/selling stocks, events 
     should get more drastic and complicated. Listed are a few examples of events:
       \begin{itemize} 
          \item A very sunny day on the island (increases value of stocks in sunscreen stocks, hotel resort stocks, electricity
                   company, etc. Simultaneously decreases value of umbrella companies, etc.)
          \item A rare shark is seen off the coast of Bot Island 
                   (increases value of tourism company stock, decreases the value of the surfing company stock)
          \item A massive storm hits Bot Island (simulates a crash in the market)
          \item Bot Island is named the best island of the year (huge jump in all stocks)
          \item Bot Island gets huge amounts of rainfall (increases value of umbrella company,
                   decreases value of sunscreen, tourism, hotel resort stocks)
      \end{itemize}
      \subsubsection{Sticking points:}
      \begin{itemize}
        \item Students can discuss their strategies amongst each other
        \item Ensure that all students have submitted a form, even if it is no change (to ensure that everyone has 
                 made their decisions before going to the next round)
        \item All stocks prices should move in each round, to simulate the stock market's general trend of upwards growth
        \item Students have the opportunity to sell their stocks and buy from the Bot Island Store during any round. 
        \item Instructors should make a point of talking about how their stocks are doing, since each will be
                 following a different strategy. The instructor who invested all their money in one stock should be 
                 exaggeratedly happy when the prices go up, and devastated when it goes down. The instructor
                 who evenly diversified should be a little jealous when the stock of the company the other instructor 
                 invested in rises, and be relieved when their stocks aren't too negatively affected in the event of a negative event. 
      \end{itemize}
    \subsection{Extensions}
      \begin{itemize}
        \item After every third round, students can receive dividends from their stocks, which will be added BotCash 
                 that the students can reinvest, or keep with them.
        \item Give students "balance sheets" of the companies containing only the most relevant information, 
                 like how much they return in dividends, their revenue, how many stocks they have in the market, etc
      \end{itemize}
    \subsection{Conclusion and Evaluation}
      At the end of the last round, announce that we have all gone into retirement, and we are now going to review
      and sell our stocks. Ask students how well they fared, and see if we can find who did the best. The student
      who did the best will discuss their strategy, and we will discuss as a class if we believe this strategy would
      work every time. Then, open the discussion to other students to see what kinds of strategies they used.
      Ask them if they have any opinions about how an instructor did. Once students have come to a conclusion,
      show data about how different funds fare, and emphasize how index funds produce the best results over
      time. Explain the concept of compound interest, and how it can take a small amount of money you have
      and grow it over time. Emphasize this fact by displaying a compounding calculator, and showing how 
      investing earlier vs later is best (and how much money they can earn, even with seemingly small rates).
      Finally, give students the amount of cash they've earned (possibly in forms of checks if we don't have
      enough paper money), and allow them to buy what they want at the Bot Island store.
      They go home with cool stuff and a knowledge of the stock market!
      \subsubsection{Exit Questions}
      \begin{itemize}
        \item What is the stock market, and what is the best strategy for investing in the market?
      \end{itemize}
\end{document}
