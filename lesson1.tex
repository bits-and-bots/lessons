\documentclass{lessonplan}

% Constants
\newcommand{\lessonNumber}{1}
\title{Lesson \lessonNumber: Getting Started with ev3 Robots and LEGO MINDSTORM programming }
\author{\linkHome}
\date{}

% Buildup
\setLessonNumber{\lessonNumber}


%Document
\begin{document}

  \maketitle

  \section{Overview}
    \subsection{Summary}
      In this lesson, the students will learn to become familiar with downloading programs to the robot and writing their names on the screen of the robot.


    \subsection{Objectives}
    \begin{itemize}
      \item Become familiar with the basics of how to download and run programs. Use the Mindstorm brick to display their name and change the LED light.  
      

    \end{itemize}
    
    
    \subsection{Key Vocabulary Terms}
    \begin{itemize}
      \item Blocks, Start Block
      \item (Intermediate) Action Blocks, Flow Blocks
      
      
    \end{itemize}
    \subsection{Materials}
      \begin{itemize}
        \item Standard ev3 brick
        \item Laptops
        \item Programming Cables
        \item Demo program
      \end{itemize}
      
      
  \section{Procedure}
    \subsection{Setup (10 minutes)}
      Turn on computers and open the programming environment. Build and turn on bots. 
    \subsection{Introduction (5 minutes)}
	Start a discussion about robots with the students. Ask what are robots, what are some kinds of robots that they know, and what can robots do. Explain that they will get an opportunity to work with their own robots. 
    \subsection{Demonstration}
      Show them their robots and explain some of the different parts of the robots. The diagram below shows a description of the buttons as well as some common status indicators on the robot. 

   \procedureImage[0.5]{1-1.png}

    Explain to students that they will be able to program the robot by creating a program on their computer, connecting their robot to the computer, and then downloading the program on their computer. You will be demonstrating this with a simple program. Have students follow the guided practice step by step. 
    \subsection{Guided Practice}
	\par Open the LEGO MINDSTORM programming enviroment. Start a new project by clicking on the plus symbol next to the Lobby tab.
	\procedureImage[0.5]{1-2.png}
	\par This will open up a blank program. Explain the differnent parts of the interface. 
	\begin{itemize}
		\item The blank part of the screen is called the "canvas"
		\item Each individual programming component is called a "block"
		\item The lower right panel is the "hardware panel", which they will use to download and run the program on the robot
	\end{itemize}

	\par Tell students that they will be using these blocks to have the robots do different things. Their very first program will be a "Hello World!" program, where they will be showing this message on the robot itself. Recreate the below program on the projector, going through step by step what the start and text blocks are and what each setting on the block does.

	\procedureImage[0.5]{1-3.png}

	\par Then, go through the process of downloading the program on the robot. Explain the difference between:
	\begin{itemize}
		\item Download setting (1)
		\item Download and run (2)
		\item Download and run selected blocks (3)
	\end{itemize}
	\procedureImage[0.5]{1-4.png}

	\par Once you have the program downloaded to the robot, show them the result and have students replicate the program. After students are able to show their messages on the screen, move on to experimenting with the LED indicator block. 

	\par Connect the LED block to the end of the display block. Show students that they are able to conenct multiple blocks together to have the robot do multiple things at once. The program should now look like this: 
	
	\procedureImage[0.5]{1-5.png}

	\par Download the program, and show students the result. Again, have students replicate this on their own robots. 
	
	\par Finally, have students use the wait block. Experiment with the placement of the wait block as well as the duration of the wait block. Students should recognize that the program is read sequentially from the start block. 

	\procedureImage[0.5]{1-6.png}

    \subsection{Extensions}
	\par Encourage students to look through the MINDSTORM programming interface and take a look at different blocks. Explain that in every session, you will be looking at most of these blocks in more detail. 

    \subsection{Conclusion and Evaluation}
      \subsubsection{Exit Questions}
      \begin{itemize}
        	\item How is the program read by the robot? Does the order matter/
	\item What do you think are some cool programs that you can make with these robots? How will you make them?
      \end{itemize}

\end{document}
