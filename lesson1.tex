\documentclass{lessonplan}

% Constants
\newcommand{\lessonNumber}{1}
\title{Lesson \lessonNumber: Moving your Robot}
\author{\linkHome}
\date{}

% Buildup
\setLessonNumber{\lessonNumber}


%Document
\begin{document}

  \maketitle

  \section{Overview}
    \subsection{Summary}
      In this lesson, the students will learn to become familiar with downloading programs to the robot and writing their names on the screen of the robot.


    \subsection{Objectives}
    \begin{itemize}
      \item Become familiar with the basics of how to download and run programs
      

    \end{itemize}
    
    
    \subsection{Key Vocabulary Terms}
    \begin{itemize}
      \item Speed
      \item (Intermediate) Motor, Rotation
      
      
    \end{itemize}
    \subsection{Materials}
      \begin{itemize}
        \item Standard ev3 brick
        \item Laptops
        \item Programming Cables
        \item Demo program
      \end{itemize}
      
      
  \section{Procedure}
    \subsection{Setup (10 minutes)}
      Turn on computers and open the programming environment. Build and turn on bots.
    \subsection{Introduction (5 minutes)}
      Explain what robots are and connect the lesson to any previously covered class concepts.
    \subsection{Demonstration}
      Play the demo program, which should move the robot in some way.
    \subsection{Guided Practice}
    Goal: Move the robot in a square shape.

    First, add a Move Steering block to your program.
    Find the Move Steering block from the catalog of blocks at the bottom of the screen:
    
    \procedureImage[0.6]{1-1.png}

    Drag it from the catalog and connect it to the Start block like it is here:
    
    \procedureImage[0.6]{1-2.png}


    Download and run your program. Watch what the Move Steering block does before you start changing anything about it.
    \par
    Before you continue, here is a picture of the Move Steering block. Below we have written what each of the different buttons controls. You will get a chance to explore all of these.

    \procedureImage[0.6]{1-3.png}
    \textcolor{magenta}{Magenta}: Method of steering control \hfill \\
    \textcolor{red}{Red}: Steering  \hfill \\
    \textcolor{blue}{Blue}: Power  \hfill \\
    \textcolor{green}{Green}: Rotations  \hfill \\
    \textcolor{orange}{Orange}: Brake at end or not  \hfill \\

    Try changing the Power setting (blue box on the diagram) on the block.
    The Power setting is what determines how fast your robot moves. Ask the students the following questions:
    \begin{itemize}
        \item What happens if you set it to 100?
        \item What happens if you set it to -100?
        \item What happens if you set it to 0?
        \item Set the power on your block to whatever speed you want.
    \end{itemize}
    \par
    Let’s try to change the steering setting (red box on the diagram) on the block
    The Steering determines which direction your robot will move.
    \begin{itemize}
        \item What happens if you set it to 100?
        \item What happens if you set it to -100?
        \item What happens if you set it to 0?
    \end{itemize}
    \par
    Figure out how to make the robot turn to the right and then move forward
    \par
    Take a few minutes to play around with some of the other settings
    \begin{itemize}
        \item What happens if you change the Brake at End setting (orange box on the diagram)?
        \item What happens if you change the Rotations to 2 or 3 (green box on the diagram)?
        \item Can you figure out how to make the robot move forward for exactly 5 seconds (hint: there’s one setting we haven’t asked you to touch yet)?
        \item  How would you make the robot move backwards and to the left at the same time using exactly one block?
    \end{itemize}

    \subsection{Extensions}
      \begin{itemize}
        \item Make the robot move forward a couple of feet.
        \item Make the robot turn around completely.
        \item Make the robot move forward again so that it returns to its starting location.	
        \item (advanced) Make your robot try to spell out the first letter in one (or more) of your group members’ names. If you have time, try to spell out a full name!
        \item (advanced) Celebrate by having the robot make a sound of your choice.
        \item (advanced) Make your robot do a figure-8 and stop in the same place it started.
        
      \end{itemize}
    \subsection{Conclusion and Evaluation}
      Give the kids a chance to show what they can make the bots do.
      \subsubsection{Exit Questions}
      \begin{itemize}
        \item When the robot turns left, which motor spins forward? Why?
      \end{itemize}

\end{document}
