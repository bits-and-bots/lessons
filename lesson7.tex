\documentclass{lessonplan}

% Constants
\newcommand{\lessonNumber}{7}
\title{Lesson \lessonNumber: The Long Arm of the Claw}
\author{\linkHome}
\date{}

% Buildup
\setLessonNumber{\lessonNumber}


%Document
\begin{document}

  \maketitle

  \section{Overview}
    \subsection{Summary}
      This lesson teaches students how to consider side-effects (open vs. close claw) and apply their techniques towards  a real-world mechanical challenge. 

    \subsection{Objectives}
    \begin{itemize}
      \item Learn how to operate claws
      \item Learn how performming side effects changes actions
      \item (Intermediate) Learn about degrees and angles
      \item (Cooperative, Very Advanced) Collaborate on code where each team cannot see the code of teams they collaborate with
    \end{itemize}
    \subsection{Key Vocabulary Terms}
    \begin{itemize}
      \item Rotations, motor
      \item (Intermediate) Degrees
      \item (Advanced) Side effects
    \end{itemize}
    \subsection{Materials}
      \begin{itemize}
        \item Masking tape
        \item Ping Pong Balls
        \item Standard ev3 brick
        \item A claw mounted on the front  of the robot (see Setup section below)
        \item Laptops
        \item Programming Cables
        \item Demo Program: Grabs a ping pong ball with a claw and moves it
      \end{itemize}
  \section{Procedure}
    \subsection{Building (xxx minutes per Robot Claw)}
    \subsection{Setup (10 minutes)}
      \begin{itemize}
        \item Build and attatch the claws (see above)
        \item Uing the masking tape, create boxes and a starting line for the robot on the ground. See the diagram below, where the robot moves from the start line to pick up a ping pong ball from the designated diamond, and drops it at the end square: \newline
              \begin{center}
        \begin{tikzpicture}
          \node [startline] (start) {Start};
          \node [pickup, above of=start, yshift=10em] (ball) {Ball};
          \node [block, left of=start, yshift=10em] (end) {End Ball};
          \path [line] (start) -- (ball);
          \path [line] (ball) -- (end);
        \end{tikzpicture}
        \end{center}

      \end{itemize}
    \subsection{Demonstration}
      Introduce with a much simpler course where the robot simply goes forward from the starting line and after a very short distance, it picks up the ball in the designated box. The design of this course is up to you, but you can use the diagram provided below.\newline
      \begin{center}
        \begin{tikzpicture}
          \node [startline] (start) {Start};
          \node [pickup, above of=start] (ball) {Ball};
          \path [line] (start) -- (ball);
        \end{tikzpicture}
        \end{center}
    \subsection{Guided Practice}
      Encourage students to attempt the shorter course from the demonstration above before the longer course, but they can move at their own pace.

      The first step to accomplish this is figuring out how to open and close the claw. First, add a move motor block to the robot. It's located here:
      \procedureImage[0.4]{motor-block-location.png}

      After the block is added to the program, check the following fields: 
      \procedureImage[0.6]{degrees.png}
      \textcolor{yellow}{Port}: This should correspond to the port the motor is plugged into on the mindstorm body. \hfill\\
      \textcolor{red}{Motor Block Input Menu}: This should be set to degrees. \hfill\\
      
      Now that the block is added, we're ready to start opening and moving the claw. Here's what each field does:

      \procedureImage[0.7]{claw-block.png}

      \textcolor{red}{Motor Block Input Menu}: This should be set to degrees (see above). We won't change this in this lesson. \hfill\\
      \textcolor{orange}{Power}: This is how fast the motor spins (how fast the claw opens and closes).\hfill\\
      \textcolor{yellow}{Degrees}: This is how many degrees the motor spins. We had the best luck having this set to 40 to 60 degrees to open the claw and -40 to -60. to close it. Too small is better than too large\footnote{If it's too large, the motor will get stuck and the program won't terminate}. \hfill\\
      \textcolor{green}{Brake at End}: If this is a check mark, the motor will suddenly stop. If it's an X, it will decelerate slowly. \hfill\\

      * If the claw is stuck in the open position at the end of a run, it will need to be manually reset to the close position. Otherwise it may not run properly.

      After the students complete the easy course, move them onto the advance course.

      The students may have trouble figuring out they need to open the claw and \textit{then} move. This is the point of the lesson so encourage them to think through this. 

    \subsection{Extensions}
      \begin{itemize}
        \item Varying obstacle courses
        \item Trade off the ping pong ball to another group's robot who has to pick it up
      \end{itemize}
    \subsection{Conclusion and Evaluation}
      Give the students chances to run through the obstacle course
      \subsubsection{Exit Questions}
      \begin{itemize}
        \item Why did you have to open the claw first?
      \end{itemize}

\end{document}
